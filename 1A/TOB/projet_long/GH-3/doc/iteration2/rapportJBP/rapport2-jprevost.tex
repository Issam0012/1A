\documentclass[frenchb]{article}
\usepackage[T1]{fontenc}
\usepackage[latin1]{inputenc}
%Pour utilisation sous unix
%\usepackage[utf8]{inputenc}
%\usepackage[utf8x]{inputenc}
\usepackage{a4wide}
\usepackage{graphicx}
\usepackage{amssymb}
\usepackage{color}
\usepackage{babel}

\begin{document}

\begin{figure}[t]
\centering
\includegraphics[width=5cm]{inp_n7.png}
\end{figure}

\title{\vspace{4cm} \textbf{Rapport individuel - it�ration 2}}
\author{Jean-Baptiste Prevost\\ }
\date{\vspace{7cm} D�partement Sciences du Num�rique - Premi�re ann�e \\
2021-2022 }

\maketitle

\newpage
\tableofcontents

\newpage
\section{Travail effectu�}
Durant cette it�ration j'ai fait:\\
\begin{itemize}
\item Cr�ation des classes en relation avec le plateau je jeu ainsi qu'a la partie jeu.
\item Interface choix de la taille du plateau de jeu.
\item Refonte g�n�ration du plateau de jeu.
\item G�n�ration d'un plateau de jeu vierge.
\item Possibilit� de changer case par case le plateau.\\
\end{itemize}

Les classes d'environnements sont termin�s mais reste cependant susceptible a des changements. 
\\\\
\indent La g�n�ration du plateau a �t� revu avec l'ajout de la possibilit� de modifier chaque case. De plus il est d�sormais possible de choisir la taille du plateau gr�ce a une nouvelle interface lors de la cr�ation du plateau.
\\\\
\indent La sauvegarde du plateau n'est pas encore fonctionnelle.
\end{document} 